\documentclass[10pt,a4paper,onecolumn]{article}

\usepackage[utf8]{inputenx}
\usepackage[T1]{fontenc}
\usepackage{lmodern}
\usepackage{listings}
\usepackage{textcomp}
\usepackage[italian]{babel}
\usepackage{amsmath}
\usepackage{booktabs}
\usepackage{graphicx}
\usepackage[font=small,labelfont=bf,labelsep=period,tableposition=top]{caption}
\usepackage{tabularx}
\usepackage{multirow}
\usepackage{booktabs}
\usepackage{longtable}
\usepackage{fancyhdr}
\usepackage{lastpage}    
\usepackage{color}

\fancyhead{}
\renewcommand{\headrulewidth}{1pt}

\fancyhead[RE,RO]{
\begin{picture}(-135,0)
	%TODO \put(-482,-14){\includegraphics[width=0.26\textwidth]{logoHead.png}} non sono molto bravo con gli strumenti grafici... qualcuno può creare un mini logo da collocare in questa posizione... l'idea è fare la scritta "Progetto di TecWeb... magari a destra di un simbolo che possa rappresentare internet... come... booo.. un mondo con un a rete attorno.... oppure un agglomerato dei loghi di browser come quella che ho scaricato!
	\put(-475,0){\sffamily\large\leftmark}
\end{picture}
}


\cfoot{}

\fancyfoot[RO,LE]{\sffamily Pag.~\thepage{} di \pageref{LastPage}} 
\fancyfoot[RE,LO]{Il mondo di Babe}

\renewcommand{\footrulewidth}{.2pt}
\pagestyle{fancy}

\renewcommand{\sectionmark}[1]{\markboth{#1}{#1}} 

% **************************************************
% Cross-references e collegamenti ipertestuali
% **************************************************
\usepackage[hidelinks]{hyperref}
\hypersetup{%
  colorlinks=false, linktocpage=false, pdfborder={0,0,0}, pdfstartpage=1, pdfstartview=FitV,%
  urlcolor=Cyan, linkcolor=Cyan, citecolor=Black, %pagecolor=Black,
  pdfcreator={pdflatex}, pdfproducer={pdflatex with hyperref package}%
}

\definecolor{dkgreen}{rgb}{0,0.6,0}
\definecolor{gray}{rgb}{0.5,0.5,0.5}
\definecolor{mauve}{rgb}{0.58,0,0.82}
 
\lstset{ %
  language=java,                % the language of the code
  basicstyle=\footnotesize,           % the size of the fonts that are used for the code
  numbers=left,                   % where to put the line-numbers
  numberstyle=\tiny\color{gray},  % the style that is used for the line-numbers
  stepnumber=2,                   % the step between two line-numbers. If it's 1, each line 
                                  % will be numbered
  numbersep=5pt,                  % how far the line-numbers are from the code
  backgroundcolor=\color{white},      % choose the background color. You must add \usepackage{color}
  showspaces=false,               % show spaces adding particular underscores
  showstringspaces=false,         % underline spaces within strings
  showtabs=false,                 % show tabs within strings adding particular underscores
  frame=single,                   % adds a frame around the code
  rulecolor=\color{black},        % if not set, the frame-color may be changed on line-breaks within not-black text (e.g. comments (green here))
  tabsize=2,                      % sets default tabsize to 2 spaces
  captionpos=b,                   % sets the caption-position to bottom
  breaklines=true,                % sets automatic line breaking
  breakatwhitespace=false,        % sets if automatic breaks should only happen at whitespace
  title=\lstname,                   % show the filename of files included with \lstinputlisting;
                                  % also try caption instead of title
  keywordstyle=\color{blue},          % keyword style
  commentstyle=\color{dkgreen},       % comment style
  stringstyle=\color{mauve},         % string literal style
  escapeinside={\%*}{*)},            % if you want to add LaTeX within your code
  morekeywords={*,...},              % if you want to add more keywords to the set
  deletekeywords={...}              % if you want to delete keywords from the given language
}

\begin{document}
%----------------------------------------------------------
\begin{titlepage}

\begin{center}
% Upper part of the page
 
\textsc{\Large}\\[5cm]

\includegraphics[width=0.4\textwidth]{Logo.png}\\[0.3cm]  
\noindent\rule{\textwidth}{0.4pt} \\[0.3cm]
\textsc{\Huge Progetto di}\\[0.25cm]
\textsc{\Huge Tecnologie Web}\\[0.3cm]
\textsc{\Large Sito ``Il mondo di Babe''}
\noindent\rule{\textwidth}{0.4pt}\\[0.5cm]
\textit{``Sviluppare un sito accessibile secondo gli standard web''} \\[0.5cm]
\textsc{20 febbraio 2013}\\[0.5cm]
\begin{minipage}{0.4\textwidth}
\begin{flushleft} \large
\emph{Studente:}\\
Andrea Meneghinello\\
Andrea Rizzi\\
Diego Beraldin\\
Elena Zecchinato
\end{flushleft}
\end{minipage}
\begin{minipage}{0.4\textwidth}
\begin{flushright} \large
\emph{Matricola:} \\
610762\\
610761\\
booo\\
booo\\
\end{flushright}
\end{minipage}
\end{center}
\end{titlepage}
%-----------------------------------------------------------------------

\clearpage

\tableofcontents

\clearpage 

\begin{abstract}

\end{abstract}

\clearpage

\section{Progettazione concettuale}

\subsection{Analisi del dominio}
% chi sono i destinatari del sito, quali sono le loro necessita' e come le soddisfiamo
% font, colori, immagini

\subsection{Progettazione architetturale}
% quale schema organizzativo e' stato scelto, come sono state organizzate le informazioni

\subsection{Mappa del sito}
% immagine della gerarchia delle informazioni

\clearpage

\section{Accessibilità}

\section{Note su particolari pagine}

\section{Pagine dinamiche}

\subsection{Utilizzo XML}

\subsection{Utilizzo Perl}

\clearpage

\section{Norme di sviluppo}

\subsection{Norme di sviluppo XML}

\subsection{Norme di sviluppo CSS}

\clearpage

\section{Riferimenti bibliografici e materiale di consultazione online}

\end{document}
