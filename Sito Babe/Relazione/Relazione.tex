\documentclass[10pt,a4paper,onecolumn]{article}

\usepackage[utf8]{inputenx}
\usepackage[T1]{fontenc}
\usepackage{lmodern}
\usepackage{listings}
\usepackage{textcomp}
\usepackage[italian]{babel}
\usepackage{amsmath}
\usepackage{booktabs}
\usepackage{graphicx}
\usepackage[font=small,labelfont=bf,labelsep=period,tableposition=top]{caption}
\usepackage{tabularx}
\usepackage{multirow}
\usepackage{booktabs}
\usepackage{longtable}
\usepackage{fancyhdr}
\usepackage{lastpage}    
\usepackage{color}

\fancyhead{}
\renewcommand{\headrulewidth}{1pt}

\fancyhead[RE,RO]{
\begin{picture}(-135,0)
	%TODO \put(-482,-14){\includegraphics[width=0.26\textwidth]{logoHead.png}} non sono molto bravo con gli strumenti grafici... qualcuno può creare un mini logo da collocare in questa posizione... l'idea è fare la scritta "Progetto di TecWeb... magari a destra di un simbolo che possa rappresentare internet... come... booo.. un mondo con un a rete attorno.... oppure un agglomerato dei loghi di browser come quella che ho scaricato!
	\put(-475,0){\sffamily\large\leftmark}
\end{picture}
}


\cfoot{}

\fancyfoot[RO,LE]{\sffamily Pag.~\thepage{} di \pageref{LastPage}} 
\fancyfoot[RE,LO]{Il mondo di Babe}

\renewcommand{\footrulewidth}{.2pt}
\pagestyle{fancy}

\renewcommand{\sectionmark}[1]{\markboth{#1}{#1}} 

% **************************************************
% Cross-references e collegamenti ipertestuali
% **************************************************
\usepackage[hidelinks]{hyperref}
\hypersetup{%
  colorlinks=false, linktocpage=false, pdfborder={0,0,0}, pdfstartpage=1, pdfstartview=FitV,%
  urlcolor=Cyan, linkcolor=Cyan, citecolor=Black, %pagecolor=Black,
  pdfcreator={pdflatex}, pdfproducer={pdflatex with hyperref package}%
}

\definecolor{dkgreen}{rgb}{0,0.6,0}
\definecolor{gray}{rgb}{0.5,0.5,0.5}
\definecolor{mauve}{rgb}{0.58,0,0.82}
 
\lstset{ %
  language=java,                % the language of the code
  basicstyle=\footnotesize,           % the size of the fonts that are used for the code
  numbers=left,                   % where to put the line-numbers
  numberstyle=\tiny\color{gray},  % the style that is used for the line-numbers
  stepnumber=2,                   % the step between two line-numbers. If it's 1, each line 
                                  % will be numbered
  numbersep=5pt,                  % how far the line-numbers are from the code
  backgroundcolor=\color{white},      % choose the background color. You must add \usepackage{color}
  showspaces=false,               % show spaces adding particular underscores
  showstringspaces=false,         % underline spaces within strings
  showtabs=false,                 % show tabs within strings adding particular underscores
  frame=single,                   % adds a frame around the code
  rulecolor=\color{black},        % if not set, the frame-color may be changed on line-breaks within not-black text (e.g. comments (green here))
  tabsize=2,                      % sets default tabsize to 2 spaces
  captionpos=b,                   % sets the caption-position to bottom
  breaklines=true,                % sets automatic line breaking
  breakatwhitespace=false,        % sets if automatic breaks should only happen at whitespace
  title=\lstname,                   % show the filename of files included with \lstinputlisting;
                                  % also try caption instead of title
  keywordstyle=\color{blue},          % keyword style
  commentstyle=\color{dkgreen},       % comment style
  stringstyle=\color{mauve},         % string literal style
  escapeinside={\%*}{*)},            % if you want to add LaTeX within your code
  morekeywords={*,...},              % if you want to add more keywords to the set
  deletekeywords={...}              % if you want to delete keywords from the given language
}

\begin{document}
%----------------------------------------------------------
\begin{titlepage}

\begin{center}
% Upper part of the page
 
\textsc{\Large}\\[5cm]

\includegraphics[width=0.4\textwidth]{Logo.png}\\[0.3cm]  
\noindent\rule{\textwidth}{0.4pt} \\[0.3cm]
\textsc{\Huge Progetto di}\\[0.25cm]
\textsc{\Huge Tecnologie Web}\\[0.3cm]
\textsc{\Large Sito ``Il mondo di Babe''}
\noindent\rule{\textwidth}{0.4pt}\\[0.5cm]
\textit{``Sviluppare un sito accessibile secondo gli standard web''} \\[0.5cm]
\textsc{20 febbraio 2013}\\[0.5cm]
\begin{minipage}{0.4\textwidth}
\begin{flushleft} \large
\emph{Studente:}\\
Andrea Meneghinello\\
Andrea Rizzi\\
Diego Beraldin\\
Elena Zecchinato
\end{flushleft}
\end{minipage}
\begin{minipage}{0.4\textwidth}
\begin{flushright} \large
\emph{Matricola:} \\
610762\\
610761\\
booo\\
booo\\
\end{flushright}
\end{minipage}
\end{center}
\end{titlepage}
%-----------------------------------------------------------------------

\clearpage

\tableofcontents

\clearpage 

\begin{abstract}
Questo progetto consiste nella realizzazione di un sito che ha come protagonista il simpatico maialino Babe, personaggio principale del film ``Babe maialino coraggioso''.
Si tratta di un sito didattico che ha lo scopo di sensibilizzare i bambini sul tema degli animali facendoli immergere nel mondo di Babe, nella sua storia ed imparando con lui i primi rudimenti come i numeri e l'alfabeto, affinchè possano amare fin da subito gli animali ed imparare il rispetto per questi ultimi.

 
 

\end{abstract}

\clearpage

\section{Analisi dei requisiti}
% chi sono i destinatari del sito, quali sono le loro necessita' e come le soddisfiamo
% font, colori, immagini

\section{Progettazione architetturale}
% quale schema organizzativo e' stato scelto, come sono state organizzate le informazioni

\section{Mappa del sito}
% immagine della gerarchia delle informazioni

\clearpage

\section{Accessibilità}
\subsection{Il testo}
Per quanto riguarda il testo si è cercato di fare particolare attenzione alla dimensione e ai font utilizzati. Infatti, essendo un sito per bambini è bene avere un testo facilmente leggibile sia in relazione alla dimensione che ai font utilizzati.
Non si sono infatti impiegati font particolari e le dimensioni del testo sono assegnate con il parametro em che permette l'adattamento del testo in base alle caratteristiche desiderate dall'utente.  

Si è inoltre cercato di utilizzare un linguaggio il più chiaro possibile visto che il pubblico a cui ci si rivolge sono i bambini è ancor più importante essere chiari e semplici.
\subsection{I link}
Per quanto riguarda i link si è deciso di non rompere le convenzioni esterne evitando di personalizzarne l'aspetto all'interno del sito.
In questo modo l'utente non è in alcun modo indotto in errore. Considerando che il nostro target può essere potenzialmente un utente con poca esperienza si è preferito mantenere la convenzione standard sui link.

In particolare nella pagina ImparaConBabe.html sono usate delle immagini per anticipare all'utente il contenuto della pagina di destinazione del link.
Le immagini costituiscono un link insieme al testo  e perciò la zona su cui è necessario cliccare per accedere alle pagine è sufficientemente ampia da poter essere accessibile anche dalle persone che hanno difficoltà nel muovere il mouse.

All'interno del sito sono presenti dei link esterni, in questo caso si è segnalato tramite l'immagine di un mondo che il link porta all'esterno del sito.
(Stagioni.html)
\subsection{Le immagini}
Nel sito le immagini sono molto numerose, essendo infatti un sito per bambini molte cose sono spiegate con l'aiuto di immagini.

Per ogni immagine è stata valutata la necessità di inserire del testo nell'attributo alt per permettere allo screen-reader di descrivere l'immagine o di poter visualizzare il testo alternativo qualora l'immagine non si potesse vedere.

Nella pagina ImparaConBabe.html le immagini servono solo per dare all'utente un'idea del contenuto della pagina che andranno ad aprire, perciò si è deciso di non mettere il testo alternativo. 
Nella pagina Alfabeto.html si è deciso di non commentare gli alt in quanto il testo presente è sufficientemente descrittivo e la presenza di testo nell'attributo alt risulterebbe superflua.
Si è fatta una scelta analoga per la pagina dei colori, dove si è ritenuto il contenuto testuale sufficiente anche senza la presenza del testo alternativo.

Anche per la pagina dei giorni si è scelto di non inserire il testo alternativo visto che le immagini sono la rappresentazione della filastrocca e quindi descriverle sarebbe inutile.
Qualora le immagini per qualche motivo non fossero visualizzabili questo non inciderebbe in alcun modo sul contenuto della pagina.


Per la pagina dei numeri invece si è scelto di descrivere le immagini.
La ragione principale risiede nel fatto che, qualora le immagini non fossero visualizzabili il testo non sarebbe sufficiente a garantire la consistenza del contenuto in quanto si potrebbero leggere solo i numeri in lettere ma non in cifre.

Per la pagina Stagioni.html si è scelto di inserire del testo nell'attributo alt. Le immagini scelte, infatti,  rappresentano simbolicamente le stagioni perciò si è ritenuto giusto riportare un testo alternativo.
Una scelta analoga si è fatta per la pagina dei mesi, in quanto ogni immagine riportata rappresenta simbolicamente un mese.

 





\subsection{Le tabelle}
\subsection{Il colore}
\subsection{I form}



\section{Note su particolari pagine}

\section{Pagine dinamiche}

\subsection{Utilizzo XML}

\subsection{Utilizzo Perl}

\clearpage

\section{Norme di sviluppo}

\subsection{Norme di sviluppo XML}

\subsection{Norme di sviluppo CSS}

\clearpage

\section{Riferimenti bibliografici e materiale di consultazione online}

\end{document}
